% I could not find any laws stating that in a forestry 
% (skógrækt), stating that there must be surveillance
% of the species present. (https://www.althingi.is/lagas/149a/1955003.html)

% I also checked the national park laws, and found that there
% is no such law for national parks. 
% https://www.althingi.is/lagas/149a/2004047.html #Thingvellir
% https://www.althingi.is/lagas/149a/2007060.html #Vatnajokull.

\section{Introduction}

    Currently, to gain a sense of the species of birds
    present in an area, a specialist must be sent on-site
    to observe the birds.
    
    We propose to solve this problem using cutting edge IoT
    technology, machine learning, 
    % marketing,
    and top-notch engineering. 

\section{Methods}
    We would place a number of low-power connected devices 
    in the area where we wish to observe the bird behavior.

    These devices would optimally monitor bird-song, use
    machine learning to identify which species of bird produced
    the song\cite{614790}, and using NarrowBand-IoT
    \cite{ratasuk2016overview}, a low-power communications 
    technology, report the findings to a centralized location.
     
    
    

\section{Expected value}
    We forsee that this project can become a multidisciplinary
    cooperation between the computer-science deparment and the 
    engineering department. % and maybe the business students... 

    We estimate that given enough development the outcome of 
    this project will be a marketable product for forest-keepers
    anywhere.